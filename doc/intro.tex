\chapter{Introduction}

This document describes the \lname\ (\sname) and application programming 
interfaces (API) used to support the evaluation of biometric software within 
the \nistig\ \cite{nistig}.

\section{Rationale}

When evaluating software in a ``black box'' fashion many aspects
of program execution must be addressed, such as non-returning function calls,
I/O errors, and other resource requirements. In addition, solutions to common
problems should be portable across operating systems.

An evaluation consists of the testing of vendor-supplied
software that implements certain biometric algorithms, such as fingerprint
matching or face recognition. The NIST Image Group defines a test process
and API for each evaluation. Vendors implement the API in their software, which
is delivered to NIST as a software library, where common test driver is used to
call the vendor library to perform the biometric operation.
In order to support the common functionality used across all evaluations, such
as logging, file input/output, etc., a common framework is used.

Even though the \lname\ was written to support biometric software evaluations,
much of the framework can be used for any general purpose programs where data
storage and system interaction are needed. One goal of the \sname\ is to
reduce the low-level error processing (particularly with input and output)
done directly by applications. The \lname\ provides several abstractions that
are useful to applications so they can focus on the task at hand.

This document describes the \sname\ in two sections: Chapters containing
descriptions of each package as well as code examples, and reference sections
containing auto-generated API documentation.

The \sname\ is a work-in-progress, and future development will occur in areas
where the need arises for the testing programs of the \nistig.
