%
% Image API
%
\chapter{Image}
\label{chp-image}

The \namespace{Image} package maintains the classes and other information related to images
and image processing. Within the \lname, many classes refer to images, such as
when dealing with fingerprint data. Many biometric data standards supply the
actual image encoded in one of several standard formats. Applications can
retrieve the image as stored in the record, or decompressed by the Image class
into a ``raw'' format. Therefore, within the \sname, several of the common
compression formats are supported, removing the need for applications to
decompress the image directly, while maintaining access to the as-recorded
image format.

\section{The Image Namespace}
\label{sec-imagenamespace}
The \namespace{Image} namespace contains several data types used to represent
aspects of an image.  The types defined are chiefly used to retrieve common
information from images stored in an \class{Image} class (section
\ref{sec-imageclass}).  Data types in the Image namespace do not perform
any translation of scale units or sizing, as each set of attributes is copied 
directly from the image data itself when possible.

The same applies to images encapsulated in biometric records.  Although some 
biometic records have fields for image attributes like dimensions and 
resolution, the corresponding fields of an \class{Image} class are {\bf not} populated 
with their contents.  The \namespace{Image} namespace data types {\em are} used outside of 
the  namespace, such as in finger views, to retrieve image attributes stored as 
part of the biometric record.  Applications can compare those values against 
the values within the \class{Image} object, as in most cases those values are taken 
directly from the underlying image data.  See~\chpref{chp-view} for more 
information on image-based biometric records.

The \namespace{Image} namespace contains all of the \class{Image} classes that are used to
represent an image. These classes are described in the following sections.

\section{The Image Class}
\label{sec-imageclass}
The \class{Image} class is an abstract base class that defines a set of minimum 
functionality for all supported image formats.  Once an \class{Image} has been 
constructed, it may not be modified.  For any supported image format, the 
following information is required to be accessible:

\begin{itemize}
\item Original binary data
\item Compression algorithm
\item Decompressed (``raw'') format binary data (grayscale, full color)
\item Depth
\item Dimensions (width, height)
\item Resolution (horizontal, vertical)
\end{itemize}

A rudimentary implementation of generating a grayscale image is provided by the
\class{Image} class in \code{get\allowbreak Raw\allowbreak Grayscale\allowbreak 
Data()}.  This implementation calculates the luminance value Y (of YCbCr) for 
each pixel of a color image.  The resulting image always uses 8-bits to 
represent a pixel, but can return a raw image using 2 gray levels (1-bit) or 256 
gray levels (8-bit).  The 1-bit algorithm quantizes to black when the 8-bit 
color value is \ensuremath{\leq}127.  \class{Image} subclasses  may override and 
implement their own grayscale conversion methods.

Also of interest in the Image class is 
\code{value\allowbreak In\allowbreak Colorspace()}, a static function to 
convert color values between bit depths.

\section{Raw Image}
\label{sec-rawimage}
The \class{RawImage} class represents a decompressed image, or an image where
\code{get\allowbreak Raw\allowbreak Data()} would return the exact same data
as \code{get\allowbreak Data()}.  \class{RawImage} has no special implementation or additional methods.

\section{JPEG}
\label{sec-image-jpeg}
The \class{JPEG} class represents an image encoded according to the JPEG image 
standard~\cite{jpeg}.  Decompression and grayscale conversion are accomplished
via \lib{libjpeg}~\cite{libjpeg}.

As of version 8.0, \lib{libjpeg} provided a way to handle JPEG images 
existing within in-memory buffers, as opposed to on-disk files.  Because the 
Image class requires in-memory buffers, \class{JPEG} includes a JPEG memory
source manager implementation, but it is built only if a version of 
\lib{libjpeg} older than 8.0 is detected at compile-time.

\class{JPEG} provides a static function to determine whether or not a
data buffer appears to be encoded in the JPEG image standard format.  Errors
within libjpeg will be caught and rethrown as \class{Exception}~s.

\section{JPEGL}
\label{sec-image-jpegl}
Similar to \class{JPEG}, the \class{JPEGL} class performs \class{Image} class
services for lossless JPEG encoded images.  JPEGL decompression is performed
by \nbis~'s \lib{libjpegl}~\cite{nist:nbis}.

\section{JPEG2000}
\label{sec-image-jpeg2000}
The \class{JPEG2000} class provides Image class functionality to JPEG 
2000-encoded images~\cite{jpeg2000}.  The class makes an attempt to support the 
following JPEG 2000 codecs:
\begin{itemize}
\item JPEG 2000 codestream (.j2k)
\item JPEG 2000 compressed image data (.jp2)
\item JPEG 2000 interactive protocol (.jpt)
\end{itemize}
Decompression is provided by the OpenJPEG library (\lib{libopenjpeg})
~\cite{libopenjpeg}.  \class{JPEG2000} also provides a static function to test
whether or not an image appears to be JPEG 2000-encoded.

Not all information required by the \class{Image} class is present in a JPEG 
2000-encoded image.  In particular, some codecs and encoders omit the ``Display
Resolution Box.''  It is generally accepted that the resolution will be 72
pixels-per-inch when the ``Display Resolution Box'' is not present.

Errors within \lib{libopenjpeg} will be caught and rethrown as 
\class{Exception}~s.

\section{NetPBM}
\label{sec-image-netpbm}
The \class{NetPBM} class provides Image class functionality to all types of 
NetPBM formatted images, up to 48-bit depth.  This includes the following 
formats:

\begin{itemize}
\item ASCII Portable Bitmap (P1, .pbm)
\item ASCII Portable Graymap (P2, .pgm)
\item ASCII Portable Pixmap (P3, .ppm)
\item Binary Portable Bitmap (P4, .pbm)
\item Binary Portable Graymap (P5, .pgm)
\item Binary Portable Pixmap (P6, .ppm)
\end{itemize}

\class{NetPBM} provides some of its more general use parsing algorithms as
static functions for use outside of the class.  This includes ASCII to binary
pixel conversion.  A function to test for NetPBM formats is also provided.

\section{PNG}
\label{sec-image-png}
The \class{PNG} class represents an image encoded according to the PNG image 
standard~\cite{png}.  Decompression is provided by 
\lib{libpng}~\cite{libpng}.

\class{PNG} provides a static function to test whether or not an image appears
to be encoded in the PNG image standard format.  Errors within \lib{libpng}
are caught and rethrown as \class{Exception}~s.

\section{TIFF}
\label{sec-image-tiff}
The \class{TIFF} provides the ability to decompress many TIFF-encoded images.
Decompression routines are provided by \lib{libtiff}~\cite{libtiff}. Like most
other Image classes, only basic grayscale and RGB-based images are parsable. The
\class{TIFF} class will throw a \class{NotImplemented} exception in the event
that unsupported TIFF data is provided.

\section{WSQ}
\label{sec-image-wsq}
Images encoded in the WSQ-image standard~\cite{std:wsq} are represented by the 
\class{WSQ} class.  The WSQ decompressor found in \nbis~\cite{nist:nbis}, 
\lib{libwsq}, is used by this class.  The class provides a static function to determine whether or not an image appears to be encoded in the WSQ format.

Errors from the \lib{libwsq} will be displayed through \code{stderr}
and will {\bf not} be rethrown as \class{Exception}~s.
