%
% Device API
%
\chapter{Device}
\label{chp-device}
The \namespace{Device} package consists of classes, constants, and other
structures used to communicate with hardware devices. These include
smartcards that conforms to the ISO Smartcard standard~\cite{std:iso7816-4}.

\section{TLV}
The \class{TLV} class represents a single tag-length-value object as described
in~\cite{std:iso7816-4}. The data for a TLV can be represented in two manners:
\begin{itemize}
\item As a ``raw'' set of octets; this is the format used by smartcards;
\item As an object giving accessed to the parsed fields, data, and children.
\end{itemize}

Both ``constructed'' and ``primitive'' basic-encoding-rule (BER) TLV objects
are supported by the \class{TLV} class. Methods are provided to obtain the
children of a constructed BER-TLV and to obtain the data of a primitive
BER-TLV.

\section{Smartcard}
\subsection{APDU}

The \class{APDU} represents an \gls{apdu} that is sent to a card. An \gls{apdu}
object directly represents the data according to~\cite{std:iso7816-4} as all
fields of the the class are public. Applications can send an \gls{apdu} to the
card, but the more effective approach is to subclass \class{Smartcard} and wrap
\gls{apdu} communication with methods that are specific to the type of card.

\subsection{Smartcard Communication}

The \class{Smartcard} class provides generic access to a any card that is
inserted
in the system. An application on the card can be activated during construction.
Card data objects can be retrieved based on the object ID, and any \gls{apdu}
can be sent to the card.

Because communicating with a card depends on a command/response protocol,
\class{Smartcard} provides methods to retrieve the response returned by the
card. This retrieval is useful when the status words must be examined as many
commands can result in several values for each status word.

\newpage

\begin{lstlisting}[caption={Accessing a PIV smartcard}, label=lst:smc-example]
#include <iostream>
#include <be_device_smartcard.h>
#include <be_device_tlv.h>
#include <be_error_exception.h>

int main(int argc, char *argv[])
{
        std::cout << "Attempt to activate PIV: " << std::endl;
        for (int i = 0; i < 4; i++) {
                try {
                        std::cout << "\tReader " << i << ": ";
                        BE::Device::Smartcard smc(i,
                             {0xA0, 0x00, 0x00, 0x03, 0x08, 0x00, 0x00,
                              0x10, 0x00, 0x01, 0x00});
                        std::cout << "Found." << std::endl;

                        std::cout << "Get Card Capability Container: "
                            << std::endl;;
                        BE::Memory::uint8Array
                            objID{0x5C, 0x03, 0x5F, 0xC1, 0x07};
                        auto obj = smc.getDedicatedFileObject(objID);

			/* The CCC is contained within a TLV */
                        std::cout << BE::Device::TLV::stringFromTLV(obj, 1);

			/* Do something with the TLV data, which is the CCC */
			BE::Device::TLV tlv(obj);
			processCCC(tlv.getPrimitive());

                // The card responded with something other than normal
                // processing complete, catch the exception from the
                // Framework so the status words can be examined.
                } catch (BE::Device::Smartcard::APDUException &e) {
                                std::cout << "Bad response: ";
                                printf("0x%02hhX%02hhX\n",
                                    e.response.sw1, e.response.sw2);
                                std::cout << "Sent APDU: " << std::endl;
				// Dump the octets from the sent APDU
                                dumpUint8Array(e.apdu);
                } catch (BE::Error::ParameterError &e) {
                                std::cout << "Caught " << e.whatString();
                } catch (BE::Error::StrategyError &e) {
                        std::cout << "Other error: " << e.whatString();
                }
                std::cout << std::endl;
        }
        return (EXIT_SUCCESS);
}
\end{lstlisting}

The example code in \lstref{lst:smc-example} shows how to activate the PIV
smartcard and retrieve one of its data objects.
