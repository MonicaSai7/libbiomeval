%
% View APIs
%
\chapter{View}
\label{chp-view}
Within the \lname\ a View represents all the information that was derived from
an image of a biometric sample. For example, with a fingerprint image, any
minutiae that were extracted from that image, as well as the image itself,
are contained within a single View object. In many cases the image may not
be present, however the image size and other information is contained within
a biometric record, along with the derived information. A View is used to
represent these records as well.

View objects are created with information taken from a biometric data record,
an ANSI/NIST 2007 file, for example. Most record formats contain information
about the image itself, such as the resolution and size. The View object can
be used to retrieve this information. However, the data may differ from that
contained in the image itself, and applications can compare the corresponding
values taken from the Image object (when available) to those taken from the
View object.

In the case where a raw image is part of the biometric record, the View object's
related Image object will have identical size, resolution, etc. values because
the View class sets the Image attributes directly. For other image types
(e.g. JPEG) the Image object will return attribute values taken from the image
data.

\section{Finger Views}
\label{sec-fingerviews}

Finger views are objects that represent all the available information for a
specific finger as contained in one or more biometric records. For example,
an ANSI/NIST file may contain a Type-3 record (finger image) and an associated
Type-9 record (finger minutiae). A finger view object based on these two
records can be instantiated and used by an application to retrieve all the
desired information, including the source finger image. The internals of
record processing and error handling are encapsulated within the class.

The \sname\ provides several classes that are derived from a base View class,
contained within the Finger package. See~\chpref{chp-finger} for more
information
on the types associated with fingers and fingerprints. This section discusses
finger views, the classes which are derived from the general View class.
These subclasses represent specific biometric file types, such as ANSI/NIST
or INCITS/M1. In the latter case, two files must be provided when constructing
the object because INCITS finger image and finger minutiae records are defined
in two separate standards.

\subsection{ANSI/NIST Finger Views}

\lstset{language=c++}
\begin{lstlisting}[caption={Using an AN2K Finger View}, label=an2kfingerviewuse]
#include <fstream>
#include <iostream>
#include <be_finger_an2kview_fixedres.h>
using namespace std;
using namespace BiometricEvaluation;

int
main(int argc, char* argv[]) {

    Finger::AN2KViewFixedResolution *_an2kv
    try {
        _an2kv = new Finger::AN2KViewFixedResolution("type9-3.an2k",
	    TYPE_3_ID, 1);
    } catch (Error::DataError &e) {
        cerr << "Caught " << e.getInfo()  << endl;
        return (EXIT_FAILURE);
    } catch (Error::FileError& e) {
        cerr << "A file error occurred: " << e.getInfo() << endl;
        return (EXIT_FAILURE);
    }
    std::auto_ptr<Finger::AN2KView> an2kv(_an2kv);

    cout << "Image resolution is " << an2kv->getImageResolution() << endl;
    cout << "Image size is " << an2kv->getImageSize() << endl;
    cout << "Image depth is " << an2kv->getImageDepth() << endl;
    cout << "Compression is " << an2kv->getCompressionAlgorithm() << endl;
    cout << "Scan resolution is " << an2kv->getScanResolution() << endl;

    // Save the finger image to a file.
    tr1::shared_ptr<Image::Image> img = an2kv->getImage();
    if (img.get() == NULL) {
       cerr << "Image was not present." << endl;
       return (EXIT_FAILURE);
    }
    string filename = "rawimg";
    ofstream img_out(filename.c_str(), ofstream::binary);
    img_out.write((char *)&(img->getRawData()[0]),
        img->getRawData().size());
    if (img_out.good())
            cout << "\tFile: " << filename << endl;
    else {
        img_out.close();
        cerr << "Error occurred when writing " << filename << endl;
        return (EXIT_FAILURE);
    }
    img_out.close();

    // Get the finger minutiae sets. AN2K records can have more than one
    // set of minutiae for a finger.

    vector<Finger::AN2KMinutiaeDataRecord> mindata = an2kv->getMinutiaeDataRecordSet();
}
\end{lstlisting}

\subsection{ISO/INCITS Finger Views}
