%
% Text API
%
\chapter{Text}
\label{chp-text}
The \namespace{Text} package consists of functions to perform common operations
on \class{string}s and \code{char} arrays.  Many of the operations may be
considered ``trivial,'' but are used often enough within the \lname\ and
other applications that a common implementation in \sname\ is more
than warranted.  A complete listing of functions is available in the
documentation appendix for \namespacedoxyref{BiometricEvaluation::Text}{2}.

\lstref{lst:text-split} shows how to use the \code{split()} function from the
\namespace{Text} package.  \code{split()} can separate a \class{string} into
tokens delimited by a character, useful for processing comma- or
space-separated text files (such files could be produced by a
\class{LogSheet} (\secref{sec-logging}), for instance).  Here, a text file
containing metadata for an image is being parsed, perhaps to be passed to 
the \class{RawImage} constructor (\secref{sec-rawimage}).


\begin{lstlisting}[caption={Tokenizing a \class{string}}, label=lst:text-split]
/* Definition of input strings */
static const vector<string>::size_type filenameToken = 0;
static const vector<string>::size_type widthToken = 1;
static const vector<string>::size_type heightToken = 2;
static const vector<string>::size_type depthToken = 3;

/* Split the string, presumably input from a file */
string input = "/mnt/raw\\ images/1.raw 500 500 8";
vector<string> tokens = Text::split(input, ' ', true);

/* Assign the retrieved tokens */
string filename;
uint32_t width, height, depth;
try {
	filename = tokens.at(filenameToken);	/* "/mnt/raw images/1.raw" */
	width = atoi(tokens.at(widthToken).c_str());	/* "500" */
	height = atoi(tokens.at(heightToken).c_str());	/* "500" */
	depth = atoi(tokens.at(depthToken).c_str());	/* "8" */
} catch (out_of_range) {
	throw Error::FileError("Malformed input");
}
\end{lstlisting}

Notice the \code{true} parameter to \code{split()} in \lstref{lst:text-split}.
This instructs \code{split()} to not tokenize based on an escaped delimiter.
If \code{false}, the first token would be split into two at the presence of the
delimiter.

\namespace{Text} also contains functions to perform  hashing via \lib{OpenSSL}.
A two-line program that emulates the command-line \code{md5sum} program is
shown in \lstref{lst:text-digest}.  Changing the digest parameter to
\code{"sha1"} would make the program emulate \code{`openssl sha1`}.

\begin{lstlisting}[caption={\code{md5sum} via \sname}, label=lst:text-digest]
#include <cstdlib>
#include <iostream>

#include <be_io_utility.h>
#include <be_text.h>
#include <be_memory_autoarray.h>

using namespace std;
using namespace BiometricEvaluation;

int
main(
    int argc,
    char *argv[])
{
	if (argc == 0)
		return (EXIT_FAILURE);
	
	try {
	        Memory::uint8Array file = IO::Utility::readFile(argv[1]);
	        cout << Text::digest(file, file.size(), "md5") << "  " <<
		    argv[1] << endl;
	} catch (Error::Exception) {
		return (EXIT_FAILURE);
	}

        return (EXIT_SUCCESS);
}
\end{lstlisting}

