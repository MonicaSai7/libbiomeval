%
% System API
%
\chapter{System}
\label{chp-system}
The System package provides a set of functions in the that
return information about the hardware and operating system. This information
can be used by applications to determine the amount of real memory, number of
central processing units, or current load average. This information can be
used to dynamically tailor the application behavior, or simply to provide
additional information in a runtime log.

\lstref{cpucountuse} shows how an application can spawn several child
processes based on the number of CPUs and memory available. Note that this
information may not be available on all platforms, and therefore, the
application must be prepared to handle that situation.

\lstset{language=c++}
\begin{lstlisting}[caption={Using the System CPU Count Information}, label=cpucountuse]
#include <iostream>
#include <be_system.h>

using namespace BiometricEvaluation;

int
main(int argc, char* argv[]) {

    // perform some application setup ...

    uint32_t cpuCount;
    uint64_t memSize, vmSize;
    try {
        cpuCount =  System::getCPUCount();
        cpuCount--;    // subtract one CPU for the parent process
        memSize = System::getRealMemorySize();
        Process::Statistics::getMemorySizes(NULL, &vmSize, NULL, NULL, NULL);
        memSize -= vmSize;   // subtract off memory used by parent

        // Give each child a fraction of the memory
        spawnChildren(cpuCount, memSize / cpuCount);
    } catch (Error::NotImplemented) {
            cout << "Running a single process only." << endl;
    }

    // processing done by parent ...
}

\end{lstlisting}
