%
% Data Interchange API
%
\chapter{Data Interchange}
\label{chp-datainterchange}
The \namespace{DataInterchange} package consists of classes and other elements used to
process an entire biometric data record. For example, a single ANSI/NIST 
record, consisting of many smaller records (fingerprint images, latent data,
etc.) can be accessed by instantiating a single object. Classes in this
package typically use has-a relationships to classes in the Finger and other
packages that process individual biometric samples.

The design of classes in the \namespace{DataInterchange} package allows applications to
create a single object from a biometric record, such as an ANSI/NIST file.
After creating this object, the application can retrieve the needed information
(such as fingerprint images) from this object, or objects returned by methods
of the Data Interchange object. A typical example would be to retrieve all
images from the record and pass them into a function that extracts a biometric
template or some other image processing.

\section{ANSI/NIST Data Records}
\label{sec-ansinistdatarecords}

The ANSI/NIST Data Interchange package contains the classes used to represent
ANSI/NIST~\cite{std:an2k} records. One class, \class{AN2KRecord},
is used to represent the entire ANSI/NIST record. An object of this class
will contain objects of the \class{Finger} classes, as well as other packages. By
instantiating the \class{AN2KRecord} object, the application can retrieve all the
information and images contained in the ANSI/NIST record.

The \class{AN2KMinutiaeDataRecord} class represents an entire Type-9 record from an
ANSI/NIST file. However, some components of this class are represented by
classes in other packages. For example, the \class{AN2K7Minutiae} class in the \namespace{Feature}
package represents the ``standard'' format minutiae in the Type-9 record.

\lstref{an2kviewuse} shows how an application can retrieve all finger
captures (Type-4 records) from an ANSI/NIST record. Once the Views are
retrieved, the application obtains the set of minutiae records associated with
that View.

\begin{lstlisting}[caption={Retrieving ANSI/NIST Records}, label=an2kviewuse]
#include <iostream>
#include <be_error_exception.h>
#include <be_finger_an2kview_capture.h>

int
main(int argc, char* argv[])
{
    /*
     * Call the constructor that will open an existing AN2K file and
     * retrieve the first finger capture (Type-14) record.
     */
    std::auto_ptr<Finger::AN2KViewCapture> an2kv;
    try {
        an2kv.reset(new Finger::AN2KViewCapture("type9-14.an2k", 1));
    } catch (Error::DataError &e) {
        cout << "Caught " << e.getInfo()  << endl;
        return (EXIT_FAILURE);
    } catch (Error::FileError& e) {
        cout << "A file error occurred: " << e.getInfo() << endl;
        return (EXIT_FAILURE);
    }

    cout << "Get the set of minutiae data records: ";
    vector<Finger::AN2KMinutiaeDataRecord> records =
        an2kv->getMinutiaeDataRecordSet();
    cout << "There are " << records.size() << " minutiae records." << endl;

    /*
     * Get the info from the first minutiae record in the View.
     */
    DataInterchange::AN2KMinutiaeDataRecord type9 = records[0];

    /*
     * Get the "standard" set of minutiae.
     */
    Feature::AN2K7Minutiae an2k7m = type9.getAN2K7Minutiae();

    /*
     * Obtain the minutiae points, ridge counts, cores, and deltas.
     */
    Feature::MinutiaPointSet mps;
    Feature::RidgeCountItemSet rcs;
    Feature::CorePointSet cps;
    Feature::DeltaPointSet dps;
    try {
        mps = an2k7m->getMinutiaPoints();
        rcs = an2k7m->getRidgeCountItems();
        cps = an2k7m->getCores();
        dps = an2k7m->getDeltas();

    } catch (Error::DataError &e) {
        cout << "Caught " << e.getInfo()  << endl;
        return (EXIT_FAILURE);
    }

    cout << "There are " << mps.size() << " minutiae points:" << endl;
    /*
     * Print out the minutiae points.
     */
    for (int i = 0; i < mps.size(); i++) {
        printf("(%u,%u,%u)\n", mps[i].coordinate.x, mps[i].coordinate.y,
             mps[i].theta);
    }
    cout << "There are " << rcs.size() << " ridge counts:" << endl;
    for (int i = 0; i < rcs.size(); i++) {
        printf("(%u,%u,%u)\n", rcs[i].index_one, rcs[i].index_two,
        rcs[i].count);
    }
    cout << "There are " << cps.size() << " cores." << endl;
    cout << "There are " << dps.size() << " deltas." << endl;

    cout << "Fingerprint Reader: " << endl;
    try { cout << an2k7m->getOriginatingFingerprintReadingSystem() << endl; }
    catch (Error::ObjectDoesNotExist) { cout << "<Omitted>" << endl; }

    cout << "Pattern (primary): " <<
    Feature::AN2K7Minutiae::convertPatternClassification(
    an2k7m->getPatternClassificationSet().at(0)) << endl;

    return(EXIT_SUCCESS);
}
\end{lstlisting}

\lstref{an2klatentuse} shows how an application can retrieve all latent
finger images from a set of ANSI/NIST record retrieved from a \class{RecordStore}.
Using the \class{Image} object, the image's ``raw'' data can be retrieved and passed
to another function for processing. Note that the image data may be stored
in a compressed format inside the ANSI/NIST record, but is converted to raw
format by the \class{Image} object.

\begin{lstlisting}[caption={Retrieving ANSI/NIST Records}, label=an2klatentuse]
#include <be_io_recordstore.h>
#include <be_data_interchange_an2k.h>
using namespace BiometricEvaluation;

void
processImageData(uint8_t *buf, uint32_t size)
{
    :
    :
    :
    :
}

int
main(int argc, char* argv[]) {

    std::tr1::shared_ptr<IO::RecordStore> rs;
    try {
        rs = IO::RecordStore::openRecordStore(rsname, datadir, IO::READONLY);
    } catch (Error::Exception &e) {
        cerr << "Could not open record store: " << e.getInfo() << endl;
        return (EXIT_FAILURE);
    }

    /*
     * Read some AN2K records and construct the View objects.
     */
    Utility::uint8Array data;
    string key;
    while (true) {          // Loop through all records in store
        uint64_t rlen;
        try {
            rlen = rs->sequence(key, NULL);
        } catch (Error::ObjectDoesNotExist &e) {
            break;
        } catch (Error::Exception &e) {
            cout << "Failed sequence: " << e.getInfo() << endl;
            return (EXIT_FAILURE);
        }
        data.resize(rlen);
        try {
            rs->read(key, data);
            DataInterchange::AN2KRecord an2k(data);
            std::vector<Finger::AN2KViewLatent> latents = an2k.getFingerLatents();
            for (int i = 0; i < latents.size(); i++) {
                tr1::shared_ptr<Image::Image> img = latents[i].getImage();
                if (img != NULL) {
                    cout << "\tCompression: " << img->getCompressionAlgorithm() << endl;
                    cout << "\tDimensions: " << img->getDimensions() << endl;
                    cout << "\tResolution: " << img->getResolution() << endl;
                    cout << "\tDepth: " << img->getDepth() << endl;
                    processImageData(img->getRawData(), img->getRawData().size());
                }
            }
        } catch (Error::Exception &e) {
            return (EXIT_FAILURE);
        }
    }
    return(EXIT_SUCCESS);
}
\end{lstlisting}

\section{INCITS Data Records}
\label{sec-incitsdatarecords}

This INCITS class of data records covers all those record formats that are
derived from the standards defined by the InterNational Committee for
Information Technology Standards~\cite{incits}. These formats include the
ANSI-2004 Finger Minutiae Record Format~\cite{std:ansi378-2004}, the ISO
equivalent~\cite{std:iso19794-2}, and other data formats, including finger
images.

\subsection{Finger Views}
\label{sec-incitsfingerviews}
Within the \sname, finger view objects (Section \ref{sec-fingerviews}) can be 
created from a combination of finger minutiae and image records. However, it
is not necessary to have both records in order to create the view because each
record contains enough information to represent the finger (image size, for
example). However, if a view is contructed using only the minutiae record, then
the image itself will not be present. Alternatively, if a view is made from
an image record, no minutiae data would be available. It is possible to
construct a view without any information.

Listing~\ref{incitsfingerviewuse} shows an example of accessing the information
in an ANSI 378-2004 Finger Minutiae Record by creating an \class{ANSI2004View} object
from the record file.

\begin{lstlisting}[caption={INCITS Finger Views}, label=incitsfingerviewuse]
#include <be_finger_ansi2004view.h>
using namespace BiometricEvaluation;

int
main(int argc, char* argv[])
{
    /*
     * Create a finger view from a file containing an ANSI-2004 finger
     * minutiae record, and without the finger view record. Read the
     * third finger view from the file.
     */
    Finger::ANSI2004View fngv;
    try {
        fngv = Finger::ANSI2004View("fmr.ansi2004", "", 3);
    } catch (Error::DataError &e) {
        cout << "Caught " << e.getInfo()  << endl;
        exit (EXIT_FAILURE);
    } catch (Error::FileError& e) {
        cout << "A file error occurred: " << e.getInfo() << endl;
        exit (EXIT_FAILURE);
    }

    /*
     * The ANSI2004View implementation of the View::View interface.
     */
    cout << "Image resolution is " << fngv.getImageResolution() << endl;
    cout << "Image size is " << fngv.getImageSize() << endl;
    cout << "Image depth is " << fngv.getImageDepth() << endl;
    cout << "Compression is " << fngv.getCompressionAlgorithm() << endl;
    cout << "Scan resolution is " << fngv.getScanResolution() << endl;

    /*
     * Test the ANSI2004View implementation of the Finger::INCITSVIEW
     * interface.
     */
    cout << "Finger position is " << fngv.getPosition() << endl;
    cout << "Impression type is " << fngv.getImpressionType() << endl;
    cout << "Quality is " << fngv.getQuality() << endl;
    cout << "Eqpt ID is " << hex << showbase << fngv.getCaptureEquipmentID() << endl;
    cout << dec;

    Feature::INCITSMinutiae fmd = fngv.getMinutiaeData();
    cout << "Minutiae format is " << fmd.getFormat() << endl;
    Feature::MinutiaPointSet mps = fmd.getMinutiaPoints();
    cout << "There are " << mps.size() << " minutiae points:" << endl;
    for (int i = 0; i < mps.size(); i++)
        cout << mps[i];

    Feature::RidgeCountItemSet rcs = fmd.getRidgeCountItems();
    cout << "There are " << rcs.size() << " ridge count items:" << endl;
    for (int i = 0; i < rcs.size(); i++)
        cout << "\t" << rcs[i];

    Feature::CorePointSet cores = fmd.getCores();
    cout << "There are " << cores.size() << " cores:" << endl;
    for (int i = 0; i < cores.size(); i++)
        cout << "\t" << cores[i];

    Feature::DeltaPointSet deltas = fmd.getDeltas();
    cout << "There are " << deltas.size() << " deltas:" << endl;
    for (int i = 0; i < deltas.size(); i++)
        cout << "\t" << deltas[i];

    exit (EXIT_SUCCESS);
}
\end{lstlisting}
