%
% I/O API
%
\chapter{Input/Output}
\label{chp-io}

The \namespace{IO} package is used
by applications for the common types of input and output: managing stores of
data, log files, and individual file management. The goal of using the \namespace{IO} API 
is to relieve applications of the need to manage low-level I/O operations such
as file opening, writing, and error handling. Furthermore, by using the classes
defined in \namespace{IO}, the actual storage mechanism used for data can be managed 
efficiently and placed in a consistent location for all applications.

Many classes manage persistent storage within the file system,
taking care of file open and close operations, as well as error handling. When
errors do occur, exceptions are thrown, which then must be handled by the
application.

\section{Utility}

The \namespace{IO::Utility} namespace provides functions that are used to
manipulate the file system and other low-level mechanisms. These functions
can be used by applications in addition to being used by other classes 
within the Biometric Evaluation framework.
The functions in this package are used to directly manipulate objects in the
POSIX file system, or to check whether a file object exists.

\section{Record Management}

The \class{IO::RecordStore} class provides an abstraction for performing
record-oriented input and output to an underlying storage system. Each
implementation of the \class{RecordStore} provides a self-contained entity to
manage data on behalf of the application in a reliable, efficient manner.

Many biometric evaluations generate thousands of files in the form of processed
images and biometric templates, in addition to consuming large numbers of files
as input. In many file systems, managing large numbers of files in not
efficient, and leads to longer run times as well as difficulty in backing up
and processing these files outside of the actual evaluation.

The \class{RecordStore} abstraction de-couples the application from
the underlying storage, enabling the implementation of different strategies for
data management. One simple strategy is to store each record into a separate
file, reproducing what has typically been done in the evaluation software
itself. Archive files and small databases are other implementation strategies
that have been used.

Use of the \class{RecordStore} abstraction allows applications to switch
storage strategy by changing a few lines of code. Furthermore, error handling
is consistent for all strategies by the use of common exceptions.

\class{RecordStore}s provide no semantic meaning to the nature of the data that passes
through the store. Each record is an opaque object, given to the store as a
managed memory object, or pointer and data length, and is associated with a
string the which is the key.  Keys must
be unique and are associated with a single data item. Attempts to insert
multiple records with the same key result in an exception being thrown.

\lstref{lst:recordstoreuse} illustrates the use of a database
\class{RecordStore} within an application.

\begin{lstlisting}[caption={Using a \class{RecordStore}}, label=lst:recordstoreuse]
#include <be_io_dbrecstore.h>
#include <be_io_utility.h>
using namespace BiometricEvaluation;
int
main(int argc, char* argv[]) {

    std::shared_ptr<IO::RecordStore> srs;
    try {
	srs = IO::RecordStore::createRecordStore(
	    "myRecords", "My Record Store",
	    IO::RecordStore::Kind::BerkeleyDB);
    } catch (Error::Exception& e) {
	cout << "Caught " << e.whatString() << endl;
	return (EXIT_FAILURE);
    }

    try {
	Memory::uint8Array theData;
	theData = getSomeData();
	srs->insert("key1", theData);

	theData = getSomeData();
	srs->insert("key2", theData);

    } catch (Error::Exception& e) {
	cout << "Caught " << e.whatString() << endl;
	return (EXIT_FAILURE);
    }

    // Some more processing where new data for a key comes in ...
    theData = getSomeData();
    srs->replace("key1", theData);
  
    // Obtain the data for all keys and write data to a file
    while (true) {
	IO::RecordStore::Record record = srs->sequence();
	cout << "Read data for key " << record.key << " of length "
	    << record.data.size() << endl;
	IO::Utility::writeFile(record.data, record.key);
    }
    // The data for the key is no longer needed ...
    srs->remove("key1");
    return (EXIT_SUCCESS);
}
\end{lstlisting}

\section{Logging}
\label{sec-logging}

Many applications are required to log information during their processing. In
particular, the evaluation test drivers often create a log record for each
call to the software under test. There is a need for the log entries to be
consistent, yet any logging facility must be flexible in accepting the type of
data that is to be written to the log file.

The logging classes in the \namespace{IO} package provide a straight-forward
method for applications to record their progress without the need to manage the
low-level storage details. Management of the log messages to the backing store
is done within the \class{Logsheet} implementations. \class{Logsheet} specifies
the common interface to all implementations. In addition, objects of this class
can be created to provide a ``Null'' \class{Logsheet} where messages are
not saved.

A \class{Logsheet} is an output stream (subclass of \class{std::ostream}),
and therefore can handle built-in types and any class that supports streaming.
Each entry is numbered by the \class{Logsheet} class when written to the log.
A call to the
\code{new\allowbreak Entry()} method commits the current entry to the log, and
resets the write position to the beginning of the entry buffer.


In addition to streaming by using the \code{Logsheet::\allowbreak <<} operator, 
applications can directly commit an entry to the log file by calling the 
\code{write()} method, thereby not disrupting the entry that is being formed.
After an entry is committed, the entry number is automatically incremented.
\class{Logsheet} also supports the writing of debug and comment entries.
Each entry is prefixed with a letter code indicating the type.

\subsection{FileLogsheet}
\label{sec-filelogsheet}

\class{IO::FileLogsheet} uses a file to store the log messages. Access to this
file is not controlled, and therefore, if two instances of this class
are made with the same file name, the results are undefined. 
The description of the sheet is placed at the top of the file during
construction of the object. Objects of this class can be constructed with
a string containing a \verb=file://= \URL~(URL) or a simple file name.

\class{IO::FileLogCabinet} is a container of \class{FileLogsheet} where
each log file is contained within the same directory owned by this container
class.

The example code in \lstref{lst:logcabinetuse} shows how an application can use 
a \class{FileLogsheet}, contained within a \class{FileLogCabinet}, to record
operational information.

\begin{lstlisting}[caption={Using a \class{FileLogsheet} within a \class{FileLogCabinet}}, label=lst:logcabinetuse]
include <iostream>
#include <be_io_filelogcabinet.h>

using namespace std;
using namespace BiometricEvaluation;

int
main(int argc, char* argv[]) {
    IO::FileLogCabinet *lc;
    try {
        lc = new IO::FileLogCabinet("MyLogCabinet", "A Log Cabinet");
    } catch (Error::ObjectExists &e) {
        cout << "The Log Cabinet already exists." << endl;
        return (EXIT_FAILURE);
    } catch (Error::StrategyError& e) {
        cout << "Caught " << e.whatString() << endl;
        return (EXIT_FAILURE);
    }
    std::unique_ptr<IO::FileLogCabinet> alc(lc);
    std::shared_ptr<IO::Logsheet> ls;
    try {
        ls = alc->newLogsheet("log01", "Log Sheet in Cabinet");
    } catch (Error::ObjectExists &e) {
        cout << "The log sheet already exists." << endl;
        return (EXIT_FAILURE);
    } catch (Error::StrategyError& e) {
        cout << "Caught " << e.whatString() << endl;
        return (EXIT_FAILURE);
    }
    ls->setAutoSync(true);  // Force write of every entry when finished
    int i = 10;
    *ls << "Adding an integer value " << i << " to the log." << endl;
    ls->newEntry();

    /*
     * Further processing ....
    */
    return (EXIT_SUCCESS);
}
\end{lstlisting}

\subsection{SysLogsheet}
\label{sec-syslogsheet}

The \class{SysLogsheet} is an implementation of \class{Logsheet} which writes
log entries to a system logger service. Objects of this class are created
with a URL starting with \verb=syslog://=.
When using a system logger, the URL must give the hostname of the logger as
well as the network port: \verb=syslog://node00:4315= for example. The system
logger must understand the Syslog protocol as specified in
RFC5424~\cite{rfc5425}.

Multiple instances of a \class{SysLogsheet} can be created with the same URL
with the assumption that the logging server can manage multiple incoming
message streams.

\section{Properties}
\label{sec-properties}
The \class{Properties} class is used to store simple key-value string pairs, 
and the \class{PropertiesFile} class stores the properties in a text file.
Applications can use a \class{PropertiesFile} object to manage
runtime settings that are persistent across invocations, or use a 
\class{Properties} object to store some settings in memory.

\begin{lstlisting}[caption={Using a \class{PropertiesFile} Object}, label=lst:propertiesuse]
int
main(int argc, char* argv[]) {


        /* Construct a Properties object with initialization */
        IO::PropertiesFile rwProps{"myProps", IO::Mode::ReadWrite,
            {{"One", "1"}, {"Two", "Two"}, {"Three", "3.0"}}};

        /* Set a property from an integer value */
        rwProps.setPropertyFromInteger("One", 1);

        /* Get a property as an integer value */
        try {
                int val = rwProps.getPropertyAsInteger("One");
                std::cout << "Property One is set to " << val << ".\n";
        } catch (const Error::Exception&) {
                std::cout << "Could not retrieve property.\n";
        }

        /* Set a new property */
        rwProps.setProperty("Four", "Four");

        /* Overwrite a default property */
        try {
                rwProps.setProperty("One", "New Value");
        } catch (const Error::Exception&) {
                std::cout << "Failed to overwrite a default value" << std::endl;
        }
        std::string  sval = rwProps.getProperty("One");
        std::cout << "Property One is now set to " << sval << ".\n";
}
\end{lstlisting}

\section{Compressor}
\label{sec-compressor}

Support for data compression and decompression can be found in the
\lname~ through the \class{Compressor} class hierarchy.  \class{Compressor} is
an abstract base class defining several pure-virtual methods for compression
and decompression of buffers and files.  Derived classes implement these methods
and can be instantiated through the factory method in the base class.
As such, children should also be enumerated within \class{Compressor::Kind}. 
The \lname~ comes with an example, \class{GZIP}, which compresses and
decompresses the \lib{gzip} format through interaction with
\lib{zlib}~\cite{zlib}.

\begin{lstlisting}[caption={Using a \class{Compressor} Object}, label=lst:compressoruse]
shared_ptr<IO::Compressor> compressor;
Memory::uint8Array compressedBuffer, largeBuffer = /* ... */;
try {
	compressor = IO::Compressor::createCompressor(Compressor::Kind::GZIP);
	/* Overloaded for all combination of buffer and file */
	compressor->compress("largeInputFile", "compressedOutputFile");
	compressor->compress(largeBuffer, compressedBuffer);
} catch (Error::Exception &e) {
	cerr << "Could not compress (" << e.whatString() << ')' << endl;
}
\end{lstlisting}

Different \class{Compressor}s may be able to respond to options that tune their
operations.  These options (and approved values) should be well-documented
in the child class, however, a no-argument constructor of a child
\class{Compressor} should automatically set any required options to default
values.  Setting and retrieving these options is very similar to interacting
with a \class{Properties} object (see \secref{sec-properties}).

\begin{lstlisting}[caption={Setting \class{Compressor} Options}, label=lst:compressoroptions]
shared_ptr<IO::Compressor> compressor =
    IO::Compressor::createCompressor(Compressor::Kind::GZIP);

/* A large GZIP chunk size can speed operations on systems with copious RAM */
compressor->setOption(IO::GZIP::CHUNK_SIZE, 32768);

\end{lstlisting}
