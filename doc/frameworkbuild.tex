%
% This section is meant to be an appendix to the main document, and therefore
% there is no \chapter{} clause, due to the manner in which the complete
% document is built.
%
\section{Language Features}

The \lname was developed using the 2011 version of the C++ language standard.
It is not possible to subset \sname to use an earlier version of C++.

Two implementations of C++11 known to compile \sname are:
\begin{itemize}
\item GNU Compiler Collection version 4.8.5 on Linux.
\item Apple LLVM version 11.0.3 (clang-1103.0.32.59) on MacOS.
\end{itemize}

\section{The Framework Build System}

The distribution of \sname includes a set of \texttt{make} files used to build
the \sname library, as well as install the library and header files. These
\texttt{make} file use some features of the GNU make ~\cite{gnumake}
system, and therefore the GNU software must be installed on the user's system.
Future versions of \sname may use a different build system.

In order to tailor the build of the \sname library (file \texttt{libbiomeval}),
the \texttt{common/src/libbiomeval/Makefile} file needs editing. At the top of
this file are make variables for locating the header files and libraries for
NBIS, and other libraries.

The make file also sets variables that create subsets of the \sname.
\texttt{CORE} and \texttt{IO} are required as they form the basis of the
\sname. The \texttt{SOURCES} variable contains a list of variables pertaining
to the desired build of \sname.

\section{The CMake Build System}

Building the \sname using CMake~\cite{org:cmake} is possible, and provides
a simpler cross-platform build system. In the {\tt common/src/libbiomeval}
directory is a {\tt CMakeLists.txt} file that controls the build. Advantages
of using the CMake build system include auto-discovery of required software
dependencies and compilation of the test programs. Also, some dependencies
are optional, and when not found, the library will be built without some
functionality. For example, video support and MPI parallel job support are
optional under the CMake build.

To build static and shared library versions of libbiomeval, including the
subset of NBIS included with the Framework, the steps are:
\begin{enumerate}
\item Create a build directory; in this example, it will be under libbiomeval:
\begin{verbatim}
    mkdir build; cd build
\end{verbatim}
\item Run CMake using the {\tt CMakeLists.txt}:
\begin{verbatim}
    cmake ..
\end{verbatim}
\item Build the Framework:
\begin{verbatim}
     make
\end{verbatim}
\item Install static and shared libraries plus headers:
\begin{verbatim}
     make install
\end{verbatim}
\item Create an RPM on CentOS or RedHat Linux systems:
\begin{verbatim}
     make package
\end{verbatim}
\end{enumerate}

To build the debug version of the library, substitute for step 2:
\begin{verbatim}
     cmake -DCMAKE_BUILD_TYPE=Debug ..
\end{verbatim}

To use a different compiler for the MPI component (Intel, OpenMPI are among
 the supported compilers), substitute for step 2:
\begin{verbatim}
     cmake -DMPI_CXX_COMPILER=mpiicpc ..
\end{verbatim}

\section{External Software Dependencies}

The \lname is built upon several other software packages. The packages are
used for image processing, biometric data record formats, the message
passing interface~\cite{mpi}, as well as operating system and compiler
tool chains.

Other common software development libraries used by \sname are documented in
the sections that follow. Specific instructions for installing these packages
are not given here. However, in general, many systems that provide a packaging
system split the library support into two packages: One for runtime (containing
the binary library file only), and one for use when developing applications.
This second package installs the header files needed to build the \sname.

\subsection{NIST Biometric Image Software}

The NIST Biometric Image Software (NBIS)~\cite{nist:nbis} is a set of packages
used for ANSI-NIST~\cite{std:an2k}, WSQ~\cite{std:wsq} formats, and other
support.  The \sname uses NBIS to process these biometric record formats.
and contains a subset of the NBIS packages. Therefore there is no need to
install NBIS.  However, the \sname build system supports using an installed
NBIS package as an alternative.

\subsection{Video and Image Processing}

For the \namespace{Image} classes, the JPEG~\cite{libjpeg},
NBIS~\cite{nist:nbis}, OpenJPEG~\cite{libopenjpeg}, PNG~\cite{libpng}, and
TIFF~\cite{libtiff} development libraries are required.

For \namespace{Video} classes, the FFmpeg~\cite{libffmpeg} libraries are used.
When building from source, configure to build and
install shared libraries. By default, only static libraries are built.

\subsection{Cryptography}

Cryptography support is provided by the OpenSSL~\cite{openssl} library.
An example is the \texttt{openssl-devel} package on Linux systems which
provides the \texttt{libcrypto} file and associated header files for
development.

\subsection{Sqlite}

SQLite is an embedded Structured Query Language (SQL) database engine and is
used by the \class{IO::SQLiteRecordStore} class to provide an
\class{IO::RecordStore} that is backed by a SQLite database. Information
on SQLite can be found at~\cite{sqlite}.

\subsection{Berkeley Database}

The Berkeley Database {BDB}~\cite{berkeleydb} is available as both open source
and closed source commercial variants. The \sname class
\class{IO::DBRecordStore} uses the BDB software to store key/value pairs.
There are two versions of the BDB API; \sname uses version 1.85 as defined in
the original open source distribution.

\subsection{Message Passing Interface}

An implementation of the MPI specification must be installed on the user's
system before the full \sname can be built. However, the \code{MPI} package
can be optionally left out of the \sname build system, if desired.

One common implementation of MPI is OpenMPI~\cite{openmpi}, available as source
code, or binary packages. Often the MPI runtime is a separate binary package
from the MPI development software. As an example, for many Linux distributions,
an example of the runtime package is \texttt{openmpi-1.6.4-3}, while
the related development package would be \texttt{openmpi-devel-1.6.4-3}.

The location of the OpenMPI libraries may be installed in a specific location.
For example, on the CentOS-7 Linux distribution, the MPI libraries are installed
on {\tt /usr/lib64/openmpi/lib/}, but the dynamic linker configuration will
not locate those libraries, and linking of an application against the \sname
library will fail. To fix this problem create
{\tt /etc/ld.so.conf.d/openmpi.conf} with the line
{\tt /usr/lib64/openmpi/lib/}, then run the {\tt ldconfig} command (as root)
to update the dynamic linker configuration.

To build the \sname, both packages are installed. In order to run an MPI
job, only the runtime package needs to be installed on all nodes that
participate in the MPI job. Chapter~\ref{chp-mpijob} has more information on
running an MPI job.
